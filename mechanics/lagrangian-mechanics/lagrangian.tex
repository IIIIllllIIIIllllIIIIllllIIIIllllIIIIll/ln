\listfiles
\documentclass{article}

\usepackage{amsmath}
\usepackage{amssymb}

\usepackage[a4paper,margin=1in]{geometry}
\newcommand{\sgn}{\operatorname{sgn}}
\newcommand{\mb}{\mathbf}


\title{Lagrangian Mechanics}
\date{}

\begin{document}
\maketitle

Definition of particle - internal motion disregarded.

$q$ and $\dot{q}$ - generalized coordinates and velocities.

If $q$ and $\dot{q}$ are given at one time - motion for subsequent time determined. Mathematically, $\ddot{q} = \ddot{q}(q,\dot{q})$ these are the \emph{equations of motion}.

\section{Principle of Least Action}

Motion of a system is completely characterised by a function $L(q, \dot{q}, t)$. 

\begin{align}
S = \int L(q,\dot{q},t)\,dt
\end{align}

action is minimized. Derive! Replace $q(t)$ by 

\begin{align}
q(t) + \delta q(t)
\end{align}

such that $\delta q(t) = 0$ at endpoints.

\begin{align}
\delta S &= \delta \int L(q,\dot{q},t)\,dt\\
&= \int \left(\frac{\partial L}{\partial q} \delta q + \frac{\partial L}{\partial \dot{q}} \delta \dot{q}\right)\,dt
\end{align}

since $\delta{\dot{q}} = \frac{d \delta q}{d t}$, 

\begin{align}
dS = \left[\frac{\partial L}{\partial \dot{q}}\right]_a^b + \int_a^b \left(\frac{\partial L}{\partial q} - \frac{d}{dt} \frac{\partial L}{\partial \dot{q}}\right)\,dt = 0
\end{align}

Hence 

\begin{align}
\frac{\partial L}{\partial q} - \frac{d}{dt} \frac{\partial L}{\partial \dot{q}} = 0 
\end{align}

\section{Properties of L}

Two non-interacting systems $A$ and $B$ have langrangian

\begin{align}
L = L_A + L_B 
\end{align}

multiplication of a lagrangian by arbitrary factor doesn't change equations of motion.

Let

\begin{align}
L' = L + \frac{d}{dt} f(q,t)
\end{align}

Then

\begin{align}
S' &= S + \int \frac{df}{dt} dt \\
dS' &= dS
\end{align}

\section{Galilean relativity}

Random frames of reference are bad. Experimentally, however, there exist a set of inertial reference frames; space is homogenous, anisotropic. In such a frame $L$ cannot depend explicity on $\mb{r}$ or $t$, but only on $\mb{v} = \dot{\mb{r}}$. Anisotropy implies that $L = L(v^2)$ Lagrange's equation reads

\begin{align}
\frac{d}{dt} \frac{\partial L}{\partial \mb{v}} &= 0 \\
\frac{\partial L}{\partial \mb{v}} &= const.
\end{align}

since $\frac{\partial L}{\partial \mb{v}}$ is a function of only $v$, $\mb{v}$ is constant.

Galilean relativity. Physical principle, cannot be derived.

\section{Lagrangian for a free particle}

We know that $L$ depends only on $v^2$. Derive! Consider two frames $K$ and $K'$ with velocities $\mb{v}$ and 

\begin{align}
\mb{v'} = \mb{v} + \mb{\epsilon}
\end{align}

They must differ by only a total derivative of a function of coordinates and time.

\begin{align}
L(v'^2) &= L(v^2 + 2 \mb{v} \cdot \mb{\epsilon}) \\
&= L(v^2) + \frac{\partial L}{\partial v^2} 2 \mb{v} \cdot \mb{\epsilon} \\
&= L(v^2) + 2 \frac{\partial L}{\partial v^2} \frac{d}{d t}\left(\mb{r} \cdot \mb{\epsilon}\right)
\end{align}

hence $\frac{\partial L}{\partial v^2}$ is independent of $v$ and 

\begin{align}
L = \frac{1}{2} m v^2
\end{align}

now we see that this holds true for finite $\epsilon$; 

\begin{align}
L' = L + \frac{d}{dt}\left(m\mb{r} \cdot {V} + \frac{1}{2} m V^2 t\right)
\end{align}

\section{Field}

Consider a free particle with lagrangian $L = \frac{1}{2} m v^2$ interacting with another system. In general the interaction adds another term we call $-U$. Now writing the eqns,

\begin{align}
L &= \frac{1}{2} m v^2 - U \\
m \frac{d\mb{v}}{dt} &= -\frac{\partial U}{\partial \mb{r}} - \frac{d}{dt}\left(\frac{\partial U}{\partial \mb{v}}\right)
\end{align}

we define the LHS as the force $m\mb{a}$. Introduce the idea of conservative force, 

\begin{align}
U = U(\mb{r}_a, \mb{r}_b, \mb{r}_c ...)
\end{align}

Now switch to generalized coordinates q; 

\begin{align}
L = \frac{1}{2} \sum a_{ik} \dot{q}_i \dot{q}_k - U  
\end{align}

where $a_{ik} = a_{ik}(q)$

\section{Conservation Laws}

With $s$ generalized coordinates there are $2s$ variables that define the state of the system. Howerever there are some functions of these variables that remain constant through the motion of the system known as \emph{integrals} of motion. Some of them are additive.

Lagrangian of a closed system does not depend explicitly on time

\begin{align}
\frac{dL}{dt} &= \sum\frac{\partial L}{\partial q} \dot{q} + \sum \frac{\partial L}{\partial \dot{q}} \ddot{q} \\
&= \sum\dot{q}_i \frac{d}{dt} \left(\frac{\partial L}{\partial \dot{q}}\right) + \sum \frac{\partial L}{\partial \dot{q}} \ddot{q} \\
&= \sum \frac{d}{dt}\left(\dot{q}_i \frac{\partial L}{\partial \dot{q}}\right) \\
E &= \sum\dot{q}\frac{\partial L}{\partial \dot{q}} - L
\end{align}

$E$ is conserved for closed and conservative systems.

\begin{align}
\sum\dot{q}\frac{\partial L}{\partial \dot{q}} &= \sum\dot{q}\frac{\partial T}{\partial \dot{q}} \\
&= 2T
\end{align}

evidently $E = T+V$

\section{Momentum}

Homogenity of space; translate space by $\delta \mb{r}$. Then

\begin{align}
\delta L &= \sum\frac{\partial L}{\partial \mb{r}} \cdot \delta\mb{r}
\end{align}

so [use LE once] $\sum\frac{\partial L}{\partial \mb{r}} = m\mb{v}$ is conserved. $\sum F = 0$.

[generalized momenta and force, COM, p43]
[angular momentum]
[mechanical similarity]
\end{document}
