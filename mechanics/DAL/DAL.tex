\listfiles
\documentclass{article}

\usepackage{amsmath}
\usepackage{amssymb}

\usepackage[a4paper,margin=1in]{geometry}
\newcommand{\sgn}{\operatorname{sgn}}
\newcommand{\mb}{\mathbf}


\title{Basics}
\date{}

\begin{document}
\maketitle

\section{Statics}

D'Alembert's principle states that for a body in static equillibrium,

\begin{align}
\sum_i\mb{F}_i \cdot \delta\mb{r}_i = 0
\end{align}

where the $\delta\mb{r}$ are virtual displacements, that is, displacements that satisfy the equations of constraint. D'Alembert's principle can be explained by noting that the LHS is a sum of virtual work, and if it were non-zero there would be a state of lower energy.

We now restrict to constraint forces that do no virtual work; this is satisfied by 1) rigid body constraints and 2) surface constraints.

For dynamics, the appropriate modification is

\begin{align}
\sum_i(\mb{F}_i - \mb{\dot{p}}_i) \cdot \delta\mb{r}_i = 0
\end{align}


Although the sum of the virtual work is 0, we cannot say anything about the individual terms because the $\delta\mb{r}$ are not independent. Here's what's so special about using generalized coordinates: {\bf We can vary the $q_j$ independantly}.

\section{Generalized velocity}

\begin{align}
\mb{r_i} &= \mb{r_i}(q_1,...q_n, t)
\end{align}

If generalized coordinates are the $q_j$ that describe the system, generalized velocities are the $\dot{q}_j$. If we change one of the $q_j$ the change in the real position, $\mb{r}_i$ is

\begin{align}
\dot{\mb{r}}_i &= \frac{d \mb{r}_i}{d t} \\
&= \sum_j \frac{\partial \mb{r}_i}{\partial q_j} \dot{q}_j + \frac{\partial \mb{r}_i}{\partial t}
\end{align}

taking partial derivatives wrt $\dot{q}_j$ we can "cancel the dots"

\begin{align}
\frac{\partial \dot{\mb{r}}_i}{\partial \dot{q}_j} = \frac{\partial \mb{r}_i}{\partial q_j}
\end{align}

Note that this is only true if the constraints are independent of the generalized velocities.

\section{Derivation}

Recalling

\begin{align}
\mb{r_i} &= \mb{r_i}(q_1,...q_n, t)
\end{align}

first basic step of varying $\mb{r}_i$ and using chain rule:

\begin{align}
\delta\mb{r_i} &= \sum_j \frac{\partial \mb{r}_i}{\partial q_j} \delta q_j
\end{align}

Let's consider the first term in DAL,

\begin{align}
\sum_i\mb{F}_i \cdot \delta{r}_i &= \sum_i\mb{F}_i \cdot(\sum_j\frac{\partial\mb{r}_i}{\partial q_j}\delta q_j) \\
&= \sum_{i,j}\mb{F}_i \cdot \frac{\partial \mb{r}_i}{\partial q_j} \delta q_j \\
&= \sum_j Q_j \delta q_j
\end{align}

where $Q_j = \sum_i \mb{F}_i \cdot \frac{\partial \mb{r}_i}{\partial q_j}$ is the generalized force.

\section{Generalized force}

If we write $F_{i,x} = -\frac{\partial V}{\partial x}$, then clearly 

\begin{align}
Q_j = -\frac{\partial V}{\partial q_j}
\end{align}

which is another way to derive that the virtual work done by varying the $j$th generalized cordinate is $Q_j \delta q_j$

Incidentally this is one very good reason to write

\begin{align}
\mb{F} = -\mb{\nabla}V = -\frac{\partial V}{\partial \mb{r}_i}
\end{align}

\section{Momentum, Work, Kinetic Energy}

Consider now the second term

\begin{align}
\sum_i \mb{\dot{p}}_i \cdot \delta \mb{r}_i &= \sum_i m_i\mb{\ddot{r}}_i \cdot \delta \mb{r}_i \\
&= \sum_{i,j} m_i\mb{\ddot{r}}_i \cdot \frac{\partial \mb{r}_i}{\partial q_j} \delta q_j
\end{align}

the $\mb{\ddot{r}}_i \delta \mb{r}_i$ looks like it could be obtained from differentiating the kinetic energy.

\begin{align}
T &= \frac{1}{2}\sum_i m_i\dot{r}_i^2 \\
\end{align}

there we differentiate wrt $q_j$ and $\dot{q}_j$

\begin{align}
\frac{\partial T}{\partial q_j} &= \sum_i m_i\mb{\dot{r}}_i \cdot \frac{\partial \mb{\dot{r}}_i}{\partial q_j} \\
\frac{\partial T}{\partial \dot{q}_j} &= \sum_i m_i\mb{\dot{r}}_i \cdot \frac{\partial \mb{\dot{r}}_i}{\partial \dot{q}_j} \\
&= \sum_i m_i\mb{\dot{r}}_i \cdot \frac{\partial \mb{r}_i}{\partial q_j}
\end{align}

by dot-cancelling rule. Now take

\begin{align}
\frac{d}{dt} \left(\frac{\partial T}{\partial \dot{q}_j}\right) &= \sum_i m_i\mb{\ddot{r}}_i \cdot \frac{\partial \mb{r}_i}{\partial q_j} + \sum_i m_i\mb{\dot{r}}_i \cdot \frac{\partial \mb{\dot{r}}_i}{\partial q_j} \\
&= Q_j + \frac{\partial T}{\partial q_j} \\
Q_j &= \frac{d}{dt} \left(\frac{\partial \mb{\dot{r}}_i}{\partial \dot{q}_j}\right) - \frac{\partial T}{\partial q_j}
\end{align}

from here the ELE are trivial.

\end{document}
