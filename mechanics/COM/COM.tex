\listfiles
\documentclass{article}

\usepackage{amsmath}
\usepackage{amssymb}

\usepackage[a4paper,margin=1in]{geometry}
\newcommand{\mb}{\mathbf}

\title{Centre of Mass}
\date{}

\begin{document}
\maketitle

The COM frame is useful when systems have translational symmetry. This happens when $V$ is a function of $\mb{r_1 - r_2}$ (relative position) instead of on the absolute position of the particles.

\section{Classical}

\begin{align}
m_1 \frac{d^2 \mb r_1}{d t^2} + \frac{\partial V}{\partial \mb r_1} &= 0 \\
m_2 \frac{d^2 \mb r_2}{d t^2} + \frac{\partial V}{\partial \mb r_2} &= 0
\end{align}

The problem with this is that the equations are coupled, because $V = V(\mb r_1 - \mb r_2)$.

Introduce $\mb r = \mb{r_1 - r_2}$ so that $V = V(\mb r)$, and $(m_1 + m_2) \mb R = m_1\mb r_1 + m_2\mb r_2$

\begin{align}
\frac{\partial V}{\partial \mb r_1} &= \frac{\partial V}{\partial \mb r} \frac{\partial \mb r}{\partial \mb r_1} \\
&= \frac{\partial V}{\partial \mb r} \\
\frac{\partial V}{\partial \mb r_2} &= -\frac{\partial V}{\partial \mb r}
\end{align}

our equations are now

\begin{align}
m_1 \frac{d^2 \mb r_1}{d t^2} + \frac{\partial V}{\partial \mb r} &= 0 \\
m_2 \frac{d^2 \mb r_2}{d t^2} - \frac{\partial V}{\partial \mb r} &= 0
\end{align}

adding the equations, $M \frac{d^2 \mb R}{d t^2} = 0$. Momentum is conserved if there are no external forces (or, by noether's theorem, if the system has translational symmetry)

now


\begin{align}
m_1 m_2 \frac{d^2 \mb r_1}{d t^2} + m_2 \frac{\partial V}{\partial \mb r} &= 0 \\
m_2 m_1 \frac{d^2 \mb r_2}{d t^2} - m_1 \frac{\partial V}{\partial \mb r} &= 0
\end{align}

subtract,

\begin{align}
m_1 m_2 \frac{d^2 \mb r}{d t^2} + (m_1 + m_2) \frac{\partial V}{\partial \mb r} &= 0 \\
\mu \frac{d^2 \mb r}{d t^2} + \frac{\partial V}{\partial \mb r} &= 0
\end{align}

I should probably do this with matrices, change of basis, etc

\section{Quantum}

\end{document}
