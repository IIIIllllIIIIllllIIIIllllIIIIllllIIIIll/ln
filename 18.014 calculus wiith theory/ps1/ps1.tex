\listfiles
\documentclass{article}

\usepackage{amsmath}
\usepackage{amssymb}

\usepackage[a4paper,margin=1in]{geometry}

\setlength\parindent{0pt}

\title{18.014 pset 1}
\date{}

\begin{document}
\maketitle

A is Dedekind-infinite: $\exists$ $A' \subset A$ such that $A \leftrightarrow A'$

\section{Any subset of a Dedekind...}

Given $A' \subseteq A$

Show that $A$ is D-finite $\implies A'$ is D-finite

It suffices to show that $A'$ is D-infinite $\implies A$ is D-infinite

Since $A'$ is D-infinite $\exists A'' \subset A'$ and $A'' \leftrightarrow A'$

Call the bijection between $A'$ and $A''$ $f$

Construct a bijection g such that

$g(x) = f(x)$ when $x \in A'$

$g(x) = x$ otherwise

g bijects $A$ to $(A - A') \cup A'' $

hence $A$ is D-infinite

\section{Show that the following are equivalent}

1) Any injection $A \rightarrow A$ is a surjection

2) A is D-finite

3) There is no injection $i: N \rightarrow A$

First, $\lnot 2 \implies \lnot 1$. Also $\lnot 1 \implies \lnot 2$. 

Proof: By definition

Next, $\lnot 3 \implies \lnot 2$

Let $R(N) \subseteq S$ be the range of i on N. i is now a bijection $N \leftrightarrow R(N)$

Since the given $i$ is an injection, it has an inverse $i^{-1}$

Consider the injection: $x \in R(N) \rightarrow i(2 i^{-1}(x))$

This injects $R(N)$ to $R(2N)$

And because $2N \subset N, R == N, R(2N) \subset R(N)$

Hence $R$ is D-infinite and so is its parent set S.

Lastly, $\lnot 2 \implies \lnot 3$`

Call the bijection $p$

Lemma: if $X$ is D-infinite $p(X)$ is also D-infinite.

Proof: $p(X)$ bijects with $p(X') \subset p(X)$

So $p^n(A)$ is D-infinite for all n

Furthermore $p^n(A) - p^{n+1}(A)$ is nonempty because $p(p^n(A)) \subset p^n(A)$

Then map $n$ to an element in $p^n(A) - p^{n+1}(A)$. Done!

\section{Show that the union of two D-finite sets is D-finite}

$A$ finite and $B$ finite $\implies A \cup B$ D-finite

We'll show $A \cup B$ D-infinite $\implies A$ D-infinite or $B$ D-infinite

There is an injection $i$ from $N \rightarrow A \cup B$

Consider the sets $N_A = i^{-1}(A)$ and $N_B = i^{-1}(B)$.

$N_A \cup N_B = N$

We will show that one of them must be bounded.

If $N_A$ is bounded by $m_A$ and $N_B$ is bounded by $m_B$, $N$ is bounded by $\max(m_A, m_B)$. Clearly wrong.

So $S$, which is $A$ or $B$, is unbounded.

Consider the injection $j$ defined by $0 \rightarrow$ smallest element of $S$

And $n \rightarrow$ smallest element of $S$ larger than $j(n-1)$; this exists because S is unbounded

Hence $N \rightarrow S$, so S is infinite.

\section{Roots}

Use PNT

\section{Find at least 3 diophantine approximants to $\sqrt{2}$}

$1/1, 3/2, 7/5, 10/7, 17/12$

\section{$1 < s < t \implies D(x,s) \supset D(x,t)$}

$1/n^t < 1/n^s$

\section{For any irrational $x$, $|D(x,1)| = \infty$}

measure in units of $1/n$; then $m/n < x < (m+1)/n$. 

\section{If $x$ is rational then $D(x,s) < \infty$ for $s>1$}

Let $s = 1 + \epsilon$. $|\frac{a}{b} - \frac{m}{n}| = \frac{|an-bm|n^\epsilon}{n^s}$.

\begin{align*}
\frac{n^\epsilon}{bn^s} < \frac{|an-bm|n^\epsilon}{bn^s} &< \frac{1}{n^s}
\end{align*}

\begin{align*}
n^\epsilon &< b
\end{align*}

This fails when $n > b^{1/\epsilon}$; since there are a finite number of $n$ and for each $n$ a finite number of $m$, we are done

\section{Liouville numbers are transcedental}

We must prove that infinite irrationality exponent $\implies$ not a solution to polynomial equation

Solution to polynomial equation $\implies$ there exists $n, c$ such that for all $p,q$, $|x - \frac{p}{q}| > \frac{c}{q^n}$. Let $c = 1/k$, 

\begin{align*}
|x - \frac{p}{q}| > \frac{1}{kq^n} > \frac{1}{q^{n'}}
\end{align*}

for some $n'$ that depends on $q$. Hence for all $q$ there exists $n'$ such that $|x - \frac{p}{q}| > \frac{1}{q^{n'}}$ which means there are no $n'$ or above approximants with denominator $q$. $n'(q) = n + 1/\ln_k q$ so the $q-n'$ plot cuts off everything to the right of some $n'$.

\end{document}
