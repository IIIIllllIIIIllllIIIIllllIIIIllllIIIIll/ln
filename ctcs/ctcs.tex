\listfiles
\documentclass{article}

\usepackage{amsmath}
\usepackage{amssymb}
\usepackage{mathtools}

\DeclarePairedDelimiter\floor{\lfloor}{\rfloor}
\DeclareMathOperator{\Hom}{Hom}
\newcommand{\<}{\langle}
\renewcommand{\>}{\rangle}

\usepackage[a4paper,margin=1in]{geometry}

\setlength{\parindent}{0cm}
\setlength{\parskip}{1em}


\title{CTCS}
\date{}


\begin{document}
\maketitle


\section*{Sets}

\subsection*{1}

Let $h: W \rightarrow S$ be a function. Define a function $\Hom(h, T): \Hom(S, T) \to \Hom(W, T)$ by $\Hom(h, T)(g) = g \circ h$. Show that if $T$ has at least 2 elements, then $h$ is surjective iif $\Hom(h, T)$ is injective.

$\implies$: Let $h$ be surjective. We wish to show that $\Hom(h, T)$ is injective. Suppose $\Hom(h, T)(f) = \Hom(h, T)(g)$. Then

\begin{align}
\Hom(h, T)(f) &= \Hom(h, T)(g) \\
f \circ h &= g \circ h
\end{align}

Let $y \in S$ be arbitrary. Since $h$ is onto $S$, there exists $x$ such that $y = h(x)$. Then

\begin{align}
f(h(x)) &= g(h(x)) \\
f(y) &= g(y)
\end{align}

Since $y$ was arbitrary, $f = g$. Hence $\Hom(h, T)$ is injective.

$\impliedby$: Let $h$ be not surjective. We wish to show that $\Hom(h, T)$ is not injective. Since $h$ is not surjective there exists $y \in S$ such that $y \ne h(x)$ for all $x$. Let $f$ and $g$ be functions both from XXX that agree on all values in their domain except that $f(y) \ne g(y)$. Note that they are different functions. However $\Hom(h, T)(f) = \Hom(h, T)(g)$ because $g \circ h = f \circ h$. Hence $\Hom(h, T)$ is not injective.

\subsection*{2a}

Show that the mapping that takes a pair $(f: X \to S, g: X \to T)$ of functions to the function $\<f, g\>: X \to S \times T$ defined by $\<f, g\>(x) = \<f(x), g(x)\>$ is a bijection from $\Hom(X, S) \times \Hom(X, T)$ to $\Hom(X, S \times T)$.

Surjective: we show that the range of the mapping is equal to the codomain, $\Hom(X, S \times T)$. Let $h \in \Hom(X, S \times T)$ be given. Then $h: X \to S \times T$. We construct $f$ and $g$ as follows. Let $x$ be arbitrary. Then $h(x) \in S \times T$, so $h(x) = (a, b)$. Then let $f(x) = a, g(x) = b$.

Injective: Suppose $\<f, g\> = \<h, j\>$. Then for all $x$ we have

\begin{align*}
\<f, g\>(x) &= \<h, j\>(x) \\
(f(x), g(x)) &= (h(x), j(x)) \\
f(x) &= h(x) \\
g(x) &= j(x)
\end{align*}

Since this is true of all $x$ we have $f = h, g = j$

\subsection*{2b}

If you set $X = S \times T$ in (a) what does $id_{S \times T}$ correspond to under the bijection?

Let $id_{S \times T} = \<f, g\>$. Then

\begin{align*}
id_{S \times T}(s, t) &= (s, t) \\
\<f, g\>(s, t) &= (f(s), g(t)) \\
f(s) &= s \\
g(t) &= t
\end{align*}

It corresponds to $(id_S, id_T)$

\subsection*{3a}

Let $S$ and $T$ be disjoint sets. Let $V$ be a set. Let $\phi: \Hom(S, V) \times \Hom(T, V) \to \Hom(S \cup T, V)$ be the mapping that takes a pair $(f: S \to V, g: T \to V)$ to the function $\<f|g\>: S \cup T \to V$ defined by $\<f|g\>(x) = f(x)$ if $x \in S, g(x)$ if $x \in T$. Show that $\phi$ is a bijection.

Surjective: Let $S, T$ be disjoint sets. Let $h \in \Hom(S \cup T, V)$ be given. Then $h: S \cup T \to V$. Construct $f: S \to V, g: T \to V$ as follow: for each $s \in S$ let $f(s) = h(s)$, and for each $v \in V$ let $g(v) = h(v)$. Then $\<f|g\> = h$.

Injective: Suppose $\<f|g\> = \<h|j\>$. For all $s \in S$ we have

\begin{align*}
\<f|g\>(s) &= \<h|j\>(s) \\
f(s) &= h(s)
\end{align*}

Hence $f = h$. Similarly, $g = j$.

\subsection*{3b}

If you set $V = S \cup T$ in (a), what is $\phi^{-1}(id_{S \cup T})$?

Let $s \in S, t \in T$. Then $\<id_S | id_T\>(s) = s, \<id_S | id_T\>(t) = t$. Hence $\<id_S | id_T\> = id_{S \cup T}$.

\subsection*{4a}

If $P(C)$ denotes the powerset of $C$ (all subsets of $C$), then $Rel(A, B) = P(A \times B)$ denotes the set of relations from $A$ to $B$. Let $\phi: Rel(A, B) \to \Hom(A, P(B))$ be defined by $\phi(\alpha)(a) = \{b \in B | (a, b) \in \alpha \}$. Show that $\phi$ is a bijection.

Surjective: Let $h \in \Hom(A, P(B))$ be given. Then $h: A \to P(B)$ and for all $a \in A$ we have $h(a) \subseteq B$. Construct $\alpha$ as follow: $\alpha = \{(a, b) | a \in A, b \in h(a) \}$. Then

\begin{align*}
\phi(\alpha)(a) &= \{b \in B | (a, b) \in \alpha \} \\
&= \{b \in B | (a, b) \in \{(a, b) | a \in A, b \in h(a) \} \} \\
&= \{b \in B | b \in h(a) \} \\
&= h(a)
\end{align*}

as required.

Injective: suppose $\phi(X) = \phi(Y)$. Then for all $a \in A$,

\begin{align*}
\phi(X) &= \phi(Y) \\
\phi(X)(a) &= \phi(Y)(a) \\
\{b \in B | (a, b) \in X\} &= \{b \in B | (a, b) \in Y\}
\end{align*}

so for all $a \in A$, for all $b \in B$,

\begin{align*}
b \in \{b \in B | (a, b) \in X\} &\iff b \in \{b \in B | (a, b) \in Y\} \\
(a, b) \in X &\iff (a, b) \in Y
\end{align*}

Hence $X = Y$.


\subsection*{4b}

Let $A = B$. What corresponds to $\Delta_A$ under this bijection?

\begin{align*}
\phi(\Delta_A)(a) &= \{b | (a, b) \in \Delta_A \} \\
&= \{ b | a = b \} \\
&= \{a\}
\end{align*}

It corresponds to the singleton function $f(a) = \{a\}$

\subsection*{4c}

If we let $A = P(B)$ then $\phi^{-1}: \Hom(P(B), P(B)) \to Rel(P(B), B)$. What is $\phi^{-1}(id_{P(B)})$?

Let $\phi^{-1}(id_{P(B)}) = \alpha$. Let $s \subseteq B$. Then

\begin{align*}
\phi(\alpha)(s) &= \{b \in B | (s, b) \in \alpha \} \\
&= id_{P(B)}(s) \\
&= s \\
&= \{b \in B | b \in s \} \\
&= \{b \in B | (s, b) \in \{(s, b) | b \in s \} \}
\end{align*}

Hence $\alpha$ is the subset relation.

\end{document}