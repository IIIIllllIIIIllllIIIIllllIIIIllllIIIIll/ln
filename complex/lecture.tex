\listfiles
\documentclass{article}

\usepackage{amsmath}
\usepackage{amssymb}

\usepackage[a4paper,margin=1in]{geometry}
 \newcommand{\sgn}{\operatorname{sgn}}

\title{Determinant}
\date{}

\begin{document}
\maketitle

f(z) = u(z) + iv(z)

u,v real functions

[z^2 picture]

one-dimensional real functions map a number line to a number line

complex functions map planes to planes

draw a curve on the z plane and see where it maps to on the w plane

[convergence diagram]

if it converges at a, all numbers smaller than a will converge
if it diverges at b, all numbers larger than b will diverge

\begin{align}
\frac{1}{1+x^2}
\end{align}

only converges from -1 to 1

1/1-x^2 function is undefined at -1 and 1, so naturally it doesn't converge, no explanation for 1/1+x^2

in the complex plane, undefined at i and -i.

[real graph of 1/1+x^2]

[disc of convergence centered at 0 radius 1]

\section{exponential}

e^z = e^x+iy = e^xe^iy

it maps x into the magnitude and y into the angle

[z-w graph for exp]

straight verticle lines to circles of radius Re[z]
y-axis maps to unit circle

horizontal lines map to lines approaching the origin

excercise: what does
 the diagonal line map to?
uniqueness theorem: given f(x), f(z) is the only function that will agree on f(x) on R and can be written as a power series

\section{trigonometric functions}

a is a real number (for now)

e^ia = cosa + i sina
e^-ia = cos a - i sina

cos a = \frac{e^ia + e^-ia}{2}
sin a = \frac{e^ia - e^-ia}{2}

cosh a = \frac{e^a + e^-a}{2}
sinh a = \frac{e^a - e^-a}{2}

cosh(ia) = cos a

\subsection*{extension to complex numbers}

cos(x+iy) = \frac{1}{4} (e^y + e^{-y}) - \frac{1}{4}(e^y - e^{-y})...

\section{cos map}

[horizontal lines] ==(iz)==> [vertical lines] 
===(1/2 e^z)===> [circles] (in opposite directions)

R = 1/2 e^a
r = 1/2 e^-a

adding up forms an ellipse

\end{document}
