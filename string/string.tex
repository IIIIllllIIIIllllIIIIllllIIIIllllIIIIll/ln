\listfiles
\documentclass{article}

\usepackage{amsmath}
\usepackage{amssymb}

\usepackage[a4paper,margin=1in]{geometry}

\title{Strings}
\date{}


\begin{document}

Consider a string fixed at both ends we write the verticle displacement $u = u(x,t)$; applying $F=ma$ yields

\begin{align}
\frac{\partial^2 u}{\partial x^2} - \frac{1}{c^2} \frac{\partial^2 u}{\partial t^2} = 0
\end{align}

where $c^2 = T/\rho$

Subject to the boundary condition that $u(0,t) = u(L,t) = 0$. We want to solve the initial value problem; given $u(x,0)$ and $u_t(x,0)$ to find $u$. We will use two methods.

\section{D'alembert's solution}

We change variables to $\xi = x - ct$ and $\eta = x + ct$. 

\begin{align}
x &= \xi + \eta \\
ct &= \eta - \xi
\end{align}

using the chain rule,

\begin{align}
\frac{\partial}{\partial x} &= \frac{\partial}{\partial \xi} + \frac{\partial}{\partial \eta} \\
\frac{1}{c}\frac{\partial}{\partial t} &= \frac{\partial}{\partial \eta} - \frac{\partial}{\partial \xi}
\end{align}

Doing some manipulation we realise that

\begin{align}
\frac{\partial^2 u}{\partial\xi\partial\eta} = 0
\end{align}

so obviously

\begin{align}
u = f(x - ct) + g(x + ct)
\end{align}

where $f$ and $g$ are any nice (single-argument) functions. This corresponds to a wave propagating to the left and one propagating to the right.

\section{Separation of variables}

We assume that 

\begin{align}
u(x,t) = X(x)T(t)
\end{align}

We see that they satisfy ODEs

\begin{align}
\frac{d^2 X}{d x^2} + k^2X &= 0 \\
\frac{d^2 T}{d t^2} + (kc)^2T &= 0
\end{align}

the general solution for $X(x)$ is a sum of sines and cosines, but due to the boundary condition the only solutions are

\begin{align}
X(x) = A \sin(k_nx)
\end{align}

where $k_n = \frac{n\pi}{L}, n = 1,2,3...$

There are no restrictions on $T(t)$ so a solution is $u = A\sin(\frac{n\pi x}{L})\sin(\frac{n\pi vt}{L} + \phi)$

These are the familiar standing waves, with $n=1$ being the fundamental, $n=2$ first overtone (one octave higher), etc. The general solution is a sum of normal modes with their amplitudes and phases adjusted to match the initial conditions.

\end{document}
