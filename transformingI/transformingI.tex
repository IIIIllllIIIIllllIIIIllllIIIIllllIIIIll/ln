\listfiles
\documentclass{article}

\usepackage{amsmath}
\usepackage{amssymb}

\usepackage[a4paper,margin=1in]{geometry}

\title{Transforming I}
\date{}


\begin{document}
\maketitle

\section{Introduction}

I is the moment of inertia tensor. A tensor is a matrix that obeys certain transformation rules; here we derive the rules for I.

We know that the angular momentum $L$ can be written as $I\omega$. The only things we need to know to derive the transformation laws are that:

\begin{enumerate}
\item{L and $\omega$ are vectors and}
\item{In any frame there exists a matrix (tensor) I such that $L = I\omega$}
\end{enumerate}

So if we have
\begin{align}
L = I\omega
\end{align}

then in another frame
\begin{align}
L' = I'\omega'
\end{align}

where, since L and $\omega$ are vectors,
$L' = $A$L$ and likewise $\omega'$ = A$\omega$ for some transformation matrix A. Then I must transform as such

\begin{align}
I' = AIA^{-1}
\end{align}

and we note that this is a similarity transformation

\section{Example}

[diagram]

First we write down I in a basis where it is diagonal; then we only need to calculate moment of principle moments of inertia.

\begin{align}
I
=
\begin{bmatrix} \frac{1}{4} &  & \\  & \frac{1}{2} &  \\  &  & \frac{1}{4}  \end{bmatrix} MR^{2}
\end{align}

then we write the transformation matrix we use

\begin{align}
R_{\theta}
=
\begin{bmatrix} \cos\theta & -\sin\theta & \\ \sin\theta & \cos\theta &  \\  &  & 1  \end{bmatrix}
\end{align}

We then apply the transformation

\begin{align}
I' = RIR^{-1}
= [calculate]
\begin{bmatrix} \frac{1}{4} &  & \\  & \frac{1}{2} &  \\  &  & \frac{1}{4}  \end{bmatrix}
\end{align}

and as desired, $I'_{x}$ = $\frac{1}{4}\sin^2\theta + \frac{1}{2}\cos^2\theta$.

\end{document}
