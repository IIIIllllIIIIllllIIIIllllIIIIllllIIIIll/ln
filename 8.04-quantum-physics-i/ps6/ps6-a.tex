\listfiles
\documentclass{article}

\usepackage[pdftex]{graphicx}
\usepackage{amsmath}
\usepackage{amssymb}
\usepackage{hyperref}

\usepackage[a4paper,margin=1in]{geometry}

\newcommand{\half}{\frac{1}{2}}
\newcommand{\<}{\langle}
\renewcommand{\>}{\rangle}

\newcommand{\ff}{\frac{1}{\sqrt{2\pi}}}

\title{}
\date{}

\begin{document}
\maketitle

\section{Time evolution of wavefunction in box potential}

a) $|\psi|^2 = C^2 (2^2 + 3^2 + 1)\frac{a}{2} = 7a = 1, C = \frac{1}{\sqrt{7a}}$
\\
b) $c_1 = 2C\sqrt\frac{2}{a} \frac{a}{2} = 2C\sqrt\frac{a}{2}, c_2 = 3C\sqrt\frac{a}{2}, c_3 = C\sqrt\frac{a}{2}$
\\
$S = 14C^2 a/2 = 14/7a a/2 = 1$
\\
c)$\psi(t) = C(2\sin(kx)e^{-iE_2t/\hbar} + 3\sin(2kx)e^{-iE_2t/\hbar} + \sin(3kx)e^{-iE_3t/\hbar})$
\\
d) Yes
\\
e) 4:9:1 (/14)
\\
f) No
\\
g) Particle's energy not known prior to measurement.

\section{Diabatic (sudden) expansion of infinite box}

Griffiths

\section{Hermitian operators}

a)

I will use the definition that Hermitian operators are $H$ such that $H^\dagger = H$

Sum of Hermitian operators is Hermitian because $(H_1 + H_2)^\dagger = H_1^\dagger + H_2^\dagger = H_1 + H_2$

$V$ is Hermitian becauese it is multiplication by a real function; in fact it only has $\delta$ eigenvectors if it is not constant

$p^2$ is Hermitian; if we write $\psi_{xx} = -k^2\psi$ we find that $k$ must be real, otherwise we cannot normalize the state
\\
b)
$\int \psi^* H \psi dx = \int \psi^* E \psi dx = E$
\\
c)
$\hbar i \psi_p = k\psi, i\hbar \ln \psi = kp, \psi = Ce^{i\hbar p / k}$, $k$ must be real for this to be normalizable
\section{Square well centered at origin, parity}

\url{http://web.mit.edu/jlee08/Public/7.91/8.04/sol6.pdf}
\\
a)

\begin{align*}
f(x) &= Ae^{ikx} + Be^{-ikx} \\
f(-a/2) &= Ae^{-ika/2} + Be^{ika/2} = 0 \\
Ae^{-ika} + B &= 0
\end{align*}

substituting,

\begin{align*}
f(x) &= Ae^{ikx} - Ae^{-ika}e^{-ikx} \\
&= Ae^{-ika/2}(e^{ikx}e^{ika/2} - e^{-ikx}e^{-ika/2}) \\
&= C\sin k(x+a/2)
\end{align*}

Now the rest is the same, $k = \frac{n\pi}{a}$ and $C = \sqrt{\frac{2}{a}}$
\\
b) 

\begin{align*}
P\sin \frac{n\pi}{a} (x+a/2) &= \sin \frac{n\pi}{a}(-x + a/2) \\
&= \sin \frac{n\pi}{a} (-x-a/2 + a) \\
&= \sin \frac{n\pi}{a} (-x-a/2) + n\pi \\
&= -\sin \frac{n\pi}{a} (x+a/2) - n\pi \\
&= -\psi(x) (-1)^n
\end{align*}
\\
c)
\begin{align*}
H\psi &= E\psi \\
PH\psi &= EP\psi \\
HP\psi &= EP\psi - [P,H]\psi \\
\end{align*}

so $[P,H]\psi = \lambda P\psi$
but obviously $[P,D] = 0$ so

\begin{align*}
[P,V]\psi &= \lambda P\psi \\
PV\psi - VP\psi &= \lambda P\psi \\
V(-x)\psi(-x) - V(x)\psi(-x) &= \lambda \psi(-x) \\
V(-x) &= V(x) + \lambda \\
&= V(-x) + 2\lambda \\
\lambda &= 0
\end{align*}

$V$ must be an even function
\end{document}
