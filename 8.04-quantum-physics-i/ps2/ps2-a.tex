\listfiles
\documentclass{article}

\usepackage[pdftex]{graphicx}
\usepackage{amsmath}
\usepackage{amssymb}

\usepackage[a4paper,margin=1in]{geometry}

\newcommand{\half}{\frac{1}{2}}
\newcommand{\<}{\langle}
\renewcommand{\>}{\rangle}

\title{}
\date{}

\begin{document}
\maketitle

\section{Parseval's Theorem}

\begin{align*}
\int_{-\infty}^{\infty} |\psi(x)|^2 dx &= \int_{-\infty}^{\infty} |\phi(k)|^2 dk \\
\end{align*}

\begin{align*}
\int_{-\infty}^{\infty} |\psi(x)|^2 dx &= \int \left(\int \frac{1}{\sqrt{2\pi}} \phi(k) e^{ikx} dk \right) \left( \int \frac{1}{\sqrt{2\pi}}\phi^*(k') e^{-ik'x} dk' \right) dx \\
&= \frac{1}{2\pi} \int\int\int \phi(k) \phi^*(k') e^{i(k-k')x} dk' dk dx \\
&= \frac{1}{2\pi} \int\int \phi(k) \phi^*(k') 2\pi\delta(k-k') dk dk' \\
&= \int \phi(k) \phi^*(k) dk
\end{align*}

To evaluate the $\int e^{ikx} dx$ integral,

\begin{align*}
\int_{-L}^L e^{ikx} dx = \frac{2\sin(kL)}{L}
\end{align*}

The central peak between the two roots closest to $x=0$ is roughly triangular, height $2L$, width $\frac{2\pi}{L}$, so the area is

\begin{align*}
\frac{1}{2} 2L \frac{2\pi}{L} = 2\pi
\end{align*}

letting $L \rightarrow \infty$,

\begin{align*}
\int_{-\infty}^\infty e^{ikx} dx = 2\pi\delta(k)
\end{align*}

\section{Fourier transform of a square wave packet}

\begin{align*}
\phi(k) &= \frac{1}{\sqrt{2\pi}} \int_{-d/2}^{d/2} \frac{1}{\sqrt d} e^{-ikx} dx \\
&= \frac{1}{\sqrt{2\pi d}} \int_{-d/2}^{d/2} e^{-ikx} dx \\
&= \frac{1}{\sqrt{2\pi d}} \frac{2\sin(\frac{dk}{2})}{k}
\end{align*}

The width of $\psi(x)$ (which can also be defined as the distance between the first two zereos) is clearly

\begin{align*}
d
\end{align*}

while the width of $\phi(k)$ is

\begin{align*}
\frac{4\pi}{d}
\end{align*}

\section{Momentum distribution due to slit and their diffraction patterns}

Let $\psi(x,y) \sim \psi_x(x)\psi_y(y)$. Then

\begin{align*}
\psi_x(x) &\sim [-d/2 < x < d/2] \\
\phi_x(k_x) &\sim \frac{2\sin(\frac{dk_x}{2})}{k_x}
\end{align*}

Where $\phi_x(k_x)$ is the probability distribution for the $x$ component of the wavevector $k_x$. For the particle to land on $x'$, the ratio momenta upon leaving the slit must be $x' : L$; this is the same as the ratio of wavenumber since $p \sim k$. Hence 

\begin{align*}
k_x = \frac{k_0 x'}{L}
\end{align*}

Also we have $k_0 = \frac{2\pi}{\lambda}$. The argument to the sin function in $\phi(k_x)$ is then

\begin{align*}
\frac{dk_x}{2} &= \frac{d (2\pi/\lambda) (x' / L)}{2} \\
&= \frac{d \pi x'}{\lambda L} \\
&= Ax'
\end{align*}

Which is the answer up to normalization.

\section{de Broglie wavelength of macroscopic objects}
\section{Gaussian wavepacket in free space}
a)
\begin{align*}
\int dx e^{-ikx} \psi(x) &\sim \int dx e^{-(ikx + \frac{x^2}{4w_0^2})} \\
&= \int dx e^{-(\frac{x}{2w} + ikw_0)^2} e^{-k^2w_0^2}\\
&= 2 \sqrt\pi w_0 e^{-k^2 w_0^2}
\end{align*}

Where

\begin{align*}
w_0 &= \frac{1}{2k_0} \\
\end{align*}

For the pile of multiplicative factors in front we can note that $\phi(k)$ must be normalized, ie have the same form as $\psi(x)$. 
\\
b)
\begin{align*}
w_0 k_0 &= \half
\end{align*}

\end{document}
