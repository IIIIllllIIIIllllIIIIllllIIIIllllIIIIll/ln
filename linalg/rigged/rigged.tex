\listfiles
\documentclass{article}

\usepackage{amsmath}
\usepackage{amssymb}
\usepackage{amsthm}

\usepackage[a4paper,margin=1in]{geometry}

\title{Rigged Hilbert Space}
\date{}

\begin{document}
\maketitle

What interesting subsets of function space can we identify?

Let's start with a basis for this space $\{\phi_i\}$. We can define $V$, the linear combination of a finite number of $\phi_i$'s:

\begin{align}
\psi = \sum_{i=1}^n c_i \phi_i
\end{align}

\section{Conjugate space}

Let $V^\times$ = set of all $f$ such that $\left< f,\psi \right> < \infty$. $V^\times$ is any linear combination of $\phi_i$'s, even very funny ones which don't converge, because $\left< f,\psi \right> < \infty$ always consists of a finite number of terms.

\section{Competeness}

One problem with $V$ is that this space isn't complete, that a series $\{\psi_i\}$ of could converge to a function that is not in $V$. If we use mean convergence and complete $V$ (find the smallest space that is complete and contains $V$) we get Hilbert space. Some other definitions of Hilbert space:

$\sum |c_n|^2 < \infty$

$h$ is square-integrable

$H \sim H^\times$

$c_n \sim n^{-1/2}$

\end{document}
