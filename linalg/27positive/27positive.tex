\listfiles
\documentclass{article}

\usepackage{amsmath}
\usepackage{amssymb}
\usepackage{amsthm}

\usepackage[a4paper,margin=1in]{geometry}

\title{Positive-definite matrices}
\date{}

\begin{document}
\maketitle

\section{Tests}

Eigenvalues $> 0$

Leading submatrices test

Pivots positive

\begin{align}
\begin{bmatrix}
a & b \\
b & c
\end{bmatrix}
\end{align}

$\lambda_1 > 0, \lambda_2 > 0$ kk work out what this means

$a > 0, ac - b^2 > 0$

$a > 0, \frac{ac - b^2}{a} > 0$

But probably the most important test / definition is

$\left<x | A x\right> > 0$

\begin{align}
\begin{bmatrix}
2 & 6 \\
6 & 18
\end{bmatrix}
\end{align}

positive semi-definite

$\lambda = 0, 20$

pivots = 2

Now let's look at $x^T Ax = 2x_1^2 + 12 x_1x_2 + 18 x_2^2 = 2(x+3y)^2$

the coefficients are $a, 2b, c$

quadratic form

The form will be positive-definite if we can factor it into a sum of squares. This shows the connection between positive-definiteness and pivots. The things that go outside the squares are the pivots. Then we have a "bowl". Otherwise we could have a saddle point, minimum in one direction and maximum in another. The important directions to look turns out to be the eigenvector directions.

For a function $f$, the matrix of second derivatives at a point is positive-definite implies it is a true minimum. The matrix of second derivatives is obviously symmetric.

If I cut the bowl at a constant, it's obviously an ellipse.

\begin{align}
\begin{bmatrix}
2 & -1 & 0 \\
-1 & 2 & -1 \\
0 & -1 & 2
\end{bmatrix}
\end{align}

dets 2,3,4
pivots 2,3/2,4/3
eigenvalues $2-\sqrt 2, 2, 2+\sqrt 2$

ellipsoid, axes in the directions of the eigenvectors

\end{document}
