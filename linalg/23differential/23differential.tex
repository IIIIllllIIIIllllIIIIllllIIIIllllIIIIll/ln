\listfiles
\documentclass{article}

\usepackage{amsmath}
\usepackage{amssymb}
\usepackage{amsthm}

\usepackage[a4paper,margin=1in]{geometry}

\title{Differential Equations and exp(At)}
\date{}

\begin{document}
\maketitle

Consider $\frac{du}{dt} = Au$

$A = \begin{bmatrix}
-1 & 1 \\
1 & -2
\end{bmatrix}$

$\lambda = 0,-3$

so for certain eigenvectors the equation becomes $\frac{du}{dt} = 0u$ and $\frac{du}{dt} = -3u$. The $0$ solution is the steady state, the $-3$ one dies out.

$u(t) = c_1 e^{\lambda_1 t}x_1 + c_2 e^{\lambda_2 t}$

compare to the formula for difference equations

\section{Cases}

Stability, $u(\infty) = 0$ when all $Re \lambda < 0$

Steady state, $\lambda_1 = 0$ and other eigenvectors have $Re \lambda < 0$

Blowup if any $Re \lambda > 0$

\section{Special case}

For $2 \times 2$ matrix:

$a + d = \lambda_1 + \lambda_2 < 0$
$det > 0$

positive determinant, negative trace.

\section{S}

Let's write the solution down in terms of $S$ and $\Lambda$. Eigenvalues uncouple them.

$\frac{du}{dt} = Au$

set $u = Av$

$S \frac{dv}{dt} = ASv$ \\
$\frac{dv}{dt} = \Lambda v$

no coupling! Just a diagonal matrix.

$v(t) = e^{\Lambda t v(0)}$ \\
$u(t) = S e^{\Lambda t} S^{-1} u(0) = e^{At} u(0)$

what do the matrix exponentials mean?

\section{Power series}

$e^{At} = I + At + \frac{(At)^2}{2!} + ...$ \\
$(I-At)^{-1} = I + At + (At)^2 + ...$

the exponential always converges but if $A$ is too big (eigenvalue > 1) the second series will blow up.

now just write $A = S\Lambda S^{-1}$, we get $e^{At} = S e^{\Lambda t} S^{-1}$

\section{Exponential of a diagonal matrix}

$e^{\Lambda t} = \begin{bmatrix}
e^{\lambda_1 t} \\
& e^{\lambda_2 t} \\
& & ...
\end{bmatrix}$

this goes to zero when all $Re \lambda < 0$

stability for differential equation: $\lambda$ in left half plane. Powers of the matrix go to zero if $\lambda$ is contained in the unit disc.

\section{Cheapskate trick}

$y'' + by' + ky = 0$
$u = [y' y], u' = \begin{bmatrix}
-b & -k \\
1 & 0
\end{bmatrix} u$

in general, coefficients in he first row, and other rows have just one $1$ among $0$'s.
\end{document}
