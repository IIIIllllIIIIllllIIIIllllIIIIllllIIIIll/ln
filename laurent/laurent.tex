\listfiles
\documentclass{article}

\usepackage{amsmath}
\usepackage{amssymb}

\usepackage[a4paper,margin=1in]{geometry}

\title{Contour Integration}
\date{}


\begin{document}
\maketitle

\section{Laurent expansion}

A Laurant expansion is a generalization of a Taylor expansion to include negative terms, 

\begin{align}
f(z) = \displaystyle\sum\limits_{n=-\infty}^{\infty} a_n(z-c)^n
\end{align}

The terms where $n$ runs negative allow us to describe a singularity occuring at $z=c$. We say $f(z)$ has a \emph{pole} at $z=c$ if there are only a finite number of these terms and we say it has an \emph{essential singularity} if there are infinitely many of such terms. For example

\begin{align}
e^{(1/z)} = 1 + \frac{1}{z} + \frac{1}{2!}\frac{1}{z^2} + ...
\end{align}

has an essential singularity at $z=0$.

The \emph{order} of a pole is the largest $n$ for which $a_{-n} \neq 0$. The \emph{residue} of $f(z)$ at $z=c$ is the coefficient $a_{-1}$ in the Laurent expansion about $z=c$. For example

\begin{align}
\frac{e^z}{z} &= \frac{1}{z}\left(1 + z + \frac{z^2}{2!} + \frac{z^3}{3!} + ...\right) \\
                &= \frac{1}{z} + 1 + \frac{1}{2!}z + \frac{1}{3!}\frac{1}{z^2} + ...
\end{align}

has a pole of order 1 at $z=0$ with residue 1.

\section{Contours}

Given a contour $\gamma$ and a function $f(z)$ we define the contour integral of $f(z)$ around $\gamma$ as

\begin{align}
\int_\gamma f(z) dz = \lim_{dz \rightarrow 0}\sum f(z) dz
\end{align}

much the same way one would define real integration. In fact a real integral can be seen as a special case of this with the contour being intervals on $\mathbb{R}$.

\section{Substitution}

In fact we can actually use some function from $\mathbb{R}$ to $\mathbb{C}$ to parameterise the contour and then evaluate it like a real integral. For example let us evaluate (a rather important integral)

\begin{align}
\int_\gamma z^n dz
\end{align}

where $\gamma$ is the unit circle going clockwise once around the origin. We can parameterize it with $z=e^{it}$ where $0 \leq t \leq 2\pi$. Then

\begin{align}
\int_\gamma z^n dz &= \int_0^{2\pi} (e^{it})^nie^{it} dt \\
                   &= i \int_0^{2\pi} e^{i(n+1)t} dt \\
                   &= \left[\frac{e^{i(n+1)t}}{(n+1)}\right]_0^{2\pi} \\
                   &= 0
\end{align}

if $n \neq -1$. If $n=-1$ then

\begin{align}
\int_\gamma z^n dz &= \int_\gamma \frac{dz}{z} \\
                   &= i \int_0^{2\pi} dt \\
                   &= 2\pi i
\end{align}

\section{Fundamental theorem of calculus}

If the contour $\gamma$ has endpoints $a,b$ and goes from $a$ to $b$ then

\begin{align}
\int_\gamma f(z) dz &= F(b) - F(a)
\end{align}

where $F'(z) = f(z)$. Clearly if $a=b$,

Theorem: The integral of a function which has an antiderivative around a closed contour is 0.

The function $log(z)$ looks like an primitive of $1/z$ but it is not continuous on the unit circle because it is multi-valued and has branch points at irritating places. Indeed with a bit of hand-waving we can use the principal value of $\log z$ to evaluate

\begin{align}
\int_\gamma \frac{dz}{z} &= \left[ \log z \right]_\gamma \\
                         &= \left[ \log |z| + i \arg z \right]_{e^0}^{e^{2\pi i}} \\
                         &= 2\pi i
\end{align}

although the endpoints are the same we see that when going once around the unit circle $\arg z$ increase by $2\pi$.

\section{Cauchy's theorem}

\section{Residue theorem}

The proof of this theorem is rather similar to that of Stoke's theorem (except you don't need to make infinite cuts) and uses the fact that a contour integral can be evaluated by cutting up the contour and integrating in the same direction around both because the integral on the common border cancels out.

We cut the contour into smaller pieces. The pieces not surrounding a singularity of course evaluate to 0. We make sure that the ones surrounding a singularity is very small, then we can use the laurent expansion around the singularity.

Then by the integration of $z^n$ we gave above, the ony terms which will contribute to something are the $n=-1$ terms. Since $\int_\gamma \frac{dz}{z} = 2\pi i$. Then clearly 

\begin{align}
\int_\gamma f(z) dz = 2\pi i \sum R_k
\end{align}

where $R_k$ is the residue at $z=k$

\end{document}
