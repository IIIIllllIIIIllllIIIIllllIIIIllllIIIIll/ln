\listfiles
\documentclass{article}

\usepackage[pdftex]{graphicx}
\usepackage{amsmath}
\usepackage{amssymb}

\usepackage[a4paper,margin=1in]{geometry}

\newcommand{\half}{\frac{1}{2}}
\newcommand{\<}{\langle}
\renewcommand{\>}{\rangle}

\title{}
\date{}

\begin{document}
\maketitle

\section{Parseval's Theorem}

\begin{align*}
\int_{-\infty}^{\infty} |\psi(x)|^2 dx &= \int_{-\infty}^{\infty} |\phi(k)|^2 dk \\
\end{align*}

\begin{align*}
\int_{-\infty}^{\infty} |\psi(x)|^2 dx &= \int \left(\int \frac{1}{\sqrt{2\pi}} \phi(k) e^{ikx} dk \right) \left( \int \frac{1}{\sqrt{2\pi}}\phi^*(k') e^{-ik'x} dk' \right) dx \\
&= \frac{1}{2\pi} \int\int\int \phi(k) \phi^*(k') e^{i(k-k')x} dk' dk dx \\
&= \frac{1}{2\pi} \int\int \phi(k) \phi^*(k') 2\pi\delta(k-k') dk dk' \\
&= \int \phi(k) \phi^*(k) dk
\end{align*}

\begin{align*}
\int_{-L}^L e^{ikx} dx = \frac{2\sin(kL)}{L}
\end{align*}

The central peak between the two roots closest to $x=0$ is roughly triangular, height $2L$, width $\frac{2\pi}{L}$, so the area is

\begin{align*}
\frac{1}{2} 2L \frac{2\pi}{L} = 2\pi
\end{align*}

letting $L \rightarrow \infty$,

\begin{align*}
\int_{-\infty}^\infty e^{ikx} dx = 2\pi\delta(k)
\end{align*}

\section{Fourier transform of a square wave packet}

\begin{align*}
\phi(k) &= \frac{1}{\sqrt{2\pi}} \int_{-d/2}^{d/2} \frac{1}{\sqrt d} e^{-ikx} dx \\
&= \frac{1}{\sqrt{2\pi d}} \int_{-d/2}^{d/2} e^{-ikx} dx \\
&= \frac{1}{\sqrt{2\pi d}} \frac{2\sin(\frac{dk}{2})}{k}
\end{align*}

Width of $\psi(x): d$

Width of $\phi(k): 4\pi/d$

\section{Momentum distribution due to slit and their diffraction patterns}

\begin{align*}
\psi(x) &\sim [-d/2 < x < d/2] \\
\phi(k_x) &\sim \frac{2\sin(\frac{dk_x}{2})}{k_x}
\end{align*}

For the particle to land on $x'$, the ratio momenta (= ratio of wavenumber) must be $x' : L$. Hence $k_x = k_0 x' / L$. Also we have $k_0 = 2\pi/\lambda$. The argument to the sin function is then

\begin{align*}
\frac{dk_x}{2} &= \frac{d (2\pi/\lambda) (x' / L)}{2} \\
&= \frac{d \pi x'}{\lambda L} \\
&= Ax'
\end{align*}

Which is the answer up to normalization.

\section{de Broglie wavelength of macroscopic objects}
\section{Gaussian wavepacket in free space}

$\int e^{-ikx - x^2 / 4w^2} = 2 \sqrt\pi w e^{-k^2 w ^2}$,
$w_0 = 1/2k_0$,
$w_0 k_0 = 1/2$

\end{document}
